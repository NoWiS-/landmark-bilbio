\section{Résumé}
  \subsection*{blah}
\begin{frame}
  \frametitle{Découpage en sous-tâches}

  \begin{block}{}
    \begin{itemize}
      \item Résolution exacte : PSPACE-Complet
      \item Relaxation : polynomial (mais très glouton, les plan sont plus long)
      \item Boucle utilisant a planificateur classique en découpant la tâche grâce aux landmarks (but intermédiaire)
    \end{itemize}
      
      \begin{thebibliography}{10}
\footnotesize

\bibitem{Hoffmann04}
Jörg Hoffmann, Julie Porteous, and Laura Sebastia.
\newblock Ordered landmarks in planning.
\newblock {\em Journal of Artificial Intelligence Research}, 22:215--278, 2004.
\end{thebibliography}

  \end{block}
\end{frame}

\begin{frame}[allowframebreaks]
  \frametitle{Heuristiques}

  \begin{block}{Comptage des landmarks ($h_L$)}
    \begin{thebibliography}{10}
\footnotesize

\bibitem{Helmert08}
Silvia Richter, Malte Helmert, and Matthias Westphal.
\newblock Landmarks revisited.
\newblock In {\em Proceedings of the 23rd AAAI Conference on Artificial
Intelligence}, pages 975--982, 2008.
\end{thebibliography}

  \end{block} 
  
  \begin{block}{Cost-partitioning (LM-A*)}
    \begin{thebibliography}{10}
\footnotesize

\bibitem{Domshlak09b}
Erez Karpas and Carmel Domshlak.
\newblock Cost-optimal planning with landmarks.
\newblock In {\em Proc. of Int. Joint. Conf. on Artificial Intelligence
({IJCAI})}, 2009.
\end{thebibliography}

  \end{block}

  \begin{block}{LM-Cut}
    \begin{thebibliography}{10}
\footnotesize

\bibitem{Domshlak09}
Malte Helmert and Carmel Domshlak.
\newblock Landmarks, critical paths and abstractions: What’s the difference
anyway.
\newblock In {\em Proceedings of the International Conference on Automated
Planning and Scheduling ICAPS-09}, pages 162--169, 2009.
\end{thebibliography}

    \begin{thebibliography}{10}
\footnotesize

\bibitem{Domshlak11}
Malte Helmert and Carmel Domshlak.
\newblock {LM}-{C}ut: Optimal planning with the landmark-cut heuristic (planner
abstract).
\newblock In {\em Proc. of the Seventh International Planning Competition (IPC
2011), Deterministic Part}, pages 103--105, 2011.
\end{thebibliography}

  \end{block}

  \begin{block}{Sound \& Complete}
    \input{cite/Zhu03}
    \begin{thebibliography}{10}
\footnotesize

\bibitem{Keyder10}
Silvia~Richter Emil~Keyder and Malte Helmert.
\newblock Sound and complete landmarks for and/or graphs.
\newblock In {\em Proc. of Euro. Conf. on Artificial Intelligence ({ECAI})},
pages 335--340, 2010.
\end{thebibliography}

  \end{block}

\end{frame}

\section{Planners}
  \subsection*{Planner}

\begin{frame}
  \frametitle{Planners utilisant des LM}
  \begin{block}{Sat}
    \begin{itemize}
      \item BJOLP : LM-A*<$h_LA$>
      \item LAMA (2008, 2011)
    \end{itemize}
  \end{block}

  \begin{block}{Opt}
    \begin{itemize}
      \item LM-cut
      \item LM-Fork
    \end{itemize}
  \end{block}
\end{frame}

\section{Quelques réf. supplémentaires...}
  \subsection*{blah}
\begin{frame}
  \begin{block}{}
    \input{cite/Richter10}  
  \end{block}
\end{frame}

