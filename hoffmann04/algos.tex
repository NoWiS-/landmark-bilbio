\section{Algorithmes}
  \subsection{Landmark}
\begin{frame}
  \frametitle{Landmark}

  \begin{block}{}
    En utilisant un graphe de planification relaxé (RPG), on construit un graphe de génération de landmark.
    On part des buts, et on remonte.
  \end{block}

  \begin{block}{}
    \begin{algorithm*}[H]
      \tiny
      \caption{ Landmark Generation Graph}
      
      \SetKwInOut{Input}{input}
      \SetKwInOut{Output}{output}
      
      \Input{$(A, O, I, G)$ a planning task}
      \Output{LGG = (N, E) où N est l'ensemble des noeuds, et E les arcs de l'arbre}
     
      $LGG = (N, E) \leftarrow (G, \emptyset)$\;
      $C \leftarrow G$\;
      \While{$C \ne \emptyset$}{
        $C' \leftarrow \emptyset$\;
        \ForEach{$(L' \in C)$ tel que $level(L') \ne \emptyset$}{
          $A \leftarrow \{(\forall a \in A)~|~L' \in add(a)~\text{et}~level(a) = level(L) - 1\}$\;
          \ForEach{$L$ tel que $(\forall a \in A)~L \in pre(a)$}{
            \uIf{$(L \in O) \notin LGG$}{
              $C' \leftarrow C' \cup \{L\}$\;
              Insérer $L$ dans LGG\;
            }
            \lIf{$L \rightarrow_{gn} L' \notin LGG$}{
              Insérer $L \rightarrow_{gn} L'$ dans LGG\;
            }
          }
        }
        $C \leftarrow C'$
      }
    \end{algorithm*}
  \end{block}
\end{frame}

\begin{frame}
  \begin{block}{}
    On obtient pas forcément des landmarks ! (à cause de du test level RPG, ça enlève les actions non présente dans le RPG).
    En fait, ça permet d'obtenir des pseudo-landmarks (il y en a plus). Pour une recheche heuristique, ça aide (mais est-ce que ça apporte vraiment plus qu'un simple RPG ?).
  \end{block}
\end{frame}

  \subsection{Ordres}

\begin{frame}
  \frametitle{$L \rightarrow_r L'$}

  \begin{block}{}
    Interférence entre 2 landmarks (en gros, des mutexes)
  \end{block}

  \alt<1>{
    \begin{block}{}
      \begin{tikzpicture}
	\SetVertexNormal[LineColor=brown]

	\Vertex[x=4,y=4]{L}
	\Vertex[x=0,y=2]{L'}
	\Vertex[x=4,y=2]{A}
	\Vertex[x=2,y=0]{B}

%	\tikzset{EdgeStyle/.style={->,relative=false,in=90,out=60}}
  \tikzset{EdgeStyle/.style = {->}}
  \Edge[label=gn](A)(B)
	\Edge[label=gn](L')(B)
	\Edge[label=Path](L)(A)
\end{tikzpicture}

    \end{block}
  }

  \alt<2>{
    \begin{block}{}
      \begin{tikzpicture}
	\SetVertexNormal[LineColor=brown]

	\Vertex[x=4,y=4]{L}
	\Vertex[x=0,y=2]{L'}
	\Vertex[x=4,y=2]{A}
	\Vertex[x=2,y=0]{B}

%	\tikzset{EdgeStyle/.style={->,relative=false,in=90,out=60}}
  \tikzset{EdgeStyle/.style = {->}}
  \Edge[label=gn](A)(B)
	\Edge[label=gn](L')(B)
	\Edge[label=Path](L)(A)
  
  \tikzset{EdgeStyle/.style = {->}}
	\Edge[label=r](L)(L')
\end{tikzpicture}

    \end{block}
  }

\end{frame}
