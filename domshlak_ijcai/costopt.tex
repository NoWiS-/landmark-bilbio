\section{Cost-Optimal}
  \subsection*{blah}

\begin{frame}
  \begin{block}{}
    \begin{thebibliography}{10}
\footnotesize

\bibitem{Domshlak09b}
Erez Karpas and Carmel Domshlak.
\newblock Cost-optimal planning with landmarks.
\newblock In {\em Proc. of Int. Joint. Conf. on Artificial Intelligence
({IJCAI})}, pages 1728--1733, 2009.
\end{thebibliography}
 
  \end{block}

  \begin{block}{}
      \begin{center}
        $h_L(s, \pi) = \sum_{\phi \in L(s, \pi)} \text{cost}(\phi)$
      \end{center}
      
      où :
    \begin{itemize}
      \item $L(s, \pi) = L \setminus \big(\text{Accepted}(s, \pi) \setminus \text{ReqAgain}(s, \pi)\big)$
      \item $\text{cost}(\phi)$, le coût minimum de l'action \emph{o} entre les actions qui produisent $\phi$ (\emph{divisé uniformément} par le nombre de Landmark que produit \emph{o})
    \end{itemize}
    \alert{Admissible, Path-dependent}
  \end{block}
  
  \begin{block}{Division uniforme ?}
    \begin{itemize}
      \proitem Facile, rapide
      \conitem Loin d'être optimal
    \end{itemize}
  \end{block}
\end{frame}

\begin{frame}
  \begin{block}{Path dependent you said ?}
    \begin{itemize}
      \item Évaluation à chaque rencontre de l'état
      \item Utilisation de la plus haute estimation
      \thusitem Multi-path dependant : Pour chaque état \emph{s}, l'ensemble des landmarks acceptés est l'intersection de ceux acceptés par tous les chemins amenant à \emph{s} (LM-A*).
    \end{itemize}
  \end{block}
\end{frame}
  
\begin{frame}
  \begin{block}{Action landmark}
    \begin{displaymath}
      h_{LA}(s, \pi) = cost(L_U(s, \pi)) + \sum_{a\in U(s, \pi)} \mathcal{C}(a)
    \end{displaymath}

    \begin{itemize}
      \item avec $U(s, \pi)$ ensemble des actions landmark depuis s non présent dans \pi
      \item avec $L_U(s, \pi) = L(s, \pi) \setminus \cup_{a \in U(s, \pi)} L(a | s, \pi)$, ensemble des landmarks atteignables immédiatement à partir de \emph{s} grâce à une action landmark.  
      \item Somme des coûts des actions landmarks + le coût $h_L$ des autres L
    \end{itemize}
  \end{block}
\end{frame}

\begin{frame}
  \begin{block}{Résultats}
    \begin{itemize}
      \proitem $LM-A*<H_{LA}$ résout plus de tâche que $A*<H_{LA}$ (moins rapide pour les petites instances)
      \item $H_{LA}$ explore en général moins de nœud que $H_{A}$, mais passe plus de temps sur chaque nœud (et prend un peu plus de temps au total).
      \item Semble aussi bien que «flexible abstraction linear abstraction strategy» (LFPA)
    \end{itemize}
  \end{block}
\end{frame}

