\section{Sound \& Complete LM}
  \subsection*{S \& C}
\begin{frame}
  \begin{block}{}
    \input{cite/Zhu03}
  \end{block}

  \begin{definition}[Causal Landmark]
    Un fluent $f$ est un causal landmark si $f \in G$ ou si $(\forall P~\text{plan solution})(\exists a \in P) f \in pre(a)$
  \end{definition}

  \begin{block}{Algo}
    \begin{itemize}
      \item Forward Propagation dans le RPG
      \item association d'un label à chaque niveau (fluent \& action) :
        \begin{itemize}
          \item Action : union des labels des préconditions
          \item Fluent : intersection des labels des producteurs plus lui-même
        \end{itemize}
      \thusitem Les labels du but sont les landmarks de $\Pi^+$
    \end{itemize}
  \end{block}
\end{frame}

\begin{frame}
  \frametitle{De nouveau landmark issu de $\Pi^m$}

  \begin{block}{}
    \begin{thebibliography}{10}
\footnotesize

\bibitem{Keyder10}
Silvia~Richter Emil~Keyder and Malte Helmert.
\newblock Sound and complete landmarks for and/or graphs.
\newblock In {\em Proc. of Euro. Conf. on Artificial Intelligence ({ECAI})},
pages 335--340, 2010.
\end{thebibliography}

  \end{block}

  \begin{block}{Problème $\Pi^m$}
    Les fluents du problèmes $\Pi^m$ sont un ensemble d'au maximum $m$ fluent du problème $\Pi$.

    Chaque action $a$ de $\Pi$ donne un ensemble d'actions dans $\Pi^m$
  \end{block}

  \begin{block}{}
    \begin{itemize}
      \item Permet la génération de conjonction de landmarks
    \end{itemize}
  \end{block}

\end{frame}

