\section{Landmark revisited}
  \subsection*{blah}

\begin{frame}
  
  \begin{block}{}
    LAMA-2008 (winner IPC2008 seq sat)
    \begin{thebibliography}{10}
\footnotesize

\bibitem{Helmert08}
Silvia Richter, Malte Helmert, and Matthias Westphal.
\newblock Landmarks revisited.
\newblock In {\em Proceedings of the 23rd AAAI Conference on Artificial
Intelligence}, pages 975--982, 2008.
\end{thebibliography}

  \end{block}

  \begin{block}{LM \& ordre}
    \begin{itemize}
      \item généré graphes de transitions de domaines SAS\up{+}
      \item introduction d'ordre naturel $L → L'$ ($L'$ est vrai avant $L$ dans les plans où $L$ est vrai. (définition plus générale)
    \end{itemize}

  \end{block}
\end{frame}

\begin{frame}
  \begin{block}{Landmark Heuristic}
    $L(s, \pi) = L \setminus \big(\text{Accepted}(s, \pi) \setminus \text{ReqAgain}(s, \pi)\big)$
    \begin{itemize}
      \item $L$ : Ensemble des landmarks
      \item $\text{Accepted}(s, \pi)$ : Ensemble des landmarks vrai dans \emph{s} avec les LM «ordonnés avant» vrai dans l'état précèdent \emph{s} (depuis plan \pi).
      \item $\text{ReqAgain}$ : Ensemble des landmark acceptés et de nouveau requis (greedy-necessary)
      \item Valeur heuristique : $|L(s, \pi)|$
      \item \alert{non admissible} (1 action peut achever plusieurs landmarks)
    \end{itemize}

    + prefered operators (les operateurs qui produisent un LM les plus proche selon un RPG)
  \end{block}
\end{frame}

\begin{frame}
  \frametitle{Résultats}

  \begin{block}{}
    \begin{itemize}
      \item Meilleure qualité
      \item Complet
      \item Moins rapide pour les petites instances (génération des LM)
    \end{itemize}
  \end{block}
\end{frame}
